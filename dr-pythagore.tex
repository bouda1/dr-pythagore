\documentclass[a4paper]{article}

\usepackage[english]{babel}
\usepackage{lmodern}
\usepackage[utf8]{inputenc}
\usepackage[T1]{fontenc}
\usepackage{graphicx}
\usepackage{listings}
\usepackage{hyperref}
\hypersetup{
  colorlinks   = true,    % Colours links instead of ugly boxes
  urlcolor     = blue,    % Colour for external hyperlinks
  linkcolor    = blue,    % Colour of internal links
  citecolor    = red      % Colour of citations
}

\newcommand{\rem}[1]{\smallskip\noindent\textbf{Remarque #1 - }}
\newcommand{\rems}[1]{\smallskip\noindent\textbf{Remarques #1 - }}
\newcommand{\Cad}{C'est-à-dire }
\newcommand{\cad}{c'est-à-dire }

\title{Découverte}
\author{D. Boucher}
\date\today
\begin{document}

\tableofcontents
\newpage

\section{Introduction}

This article details specifications of \textit{Dr Pythagore}, a software to learn about geometry proofs. The goal is an ideal, a first idea would be to work around IA and mathematics, making experiences.

\section{The parser}

We want to define objects, sets, points and \textit{play} with them. To allow that, we introduce here a parser.

\section{DPPlan}

We know it is a points set but here we see it differently. It gives relationship between elements. It is the plan that contains properties between them.

For example, if we assert that two points are equal, this assertion is set in the plan.

Relations like \textit{equality}, \textit{orthogonality}, \textit{not equality}, etc\dots are stored in that class.

An important point is dependencies between relations. For example, let's assume $A$, $B$ and $C$ are not aligned. This assertion implies others like $(AB)$ and $(AC)$ are intersecting. The idea of storing all the consequencies looks awful, they could be too many with only several points.

So, the idea would be more to be able to use them when necessary.

Let's consider an example : Let $A$, $B$ and $C$ three points not aligned.
We could then ask if lines $(AB)$ and $(AC)$ are parallel.

The algorithm follows these ideas:
\begin{itemize}
\item To build the line $(AB)$, we must have $A\neq B$. But this is true because $A, B$ and $C$ are not aligned.
\item Then we must build the line $(AC)$ and we arrive to the same result.
\item Now, we could have a rule telling that those lines are intersecting since $A,B$ and $C$ are not aligned.
\end{itemize}

The first step is that :
\begin{itemize}
\item $A$, $B$ and $C$ are defined in the plane.
\item We known $A, B, C$ not aligned. This relation is also stored in the plane.
\item We have rules connecting relations:
\begin{itemize}
\item A first one is $A,B,C$ not aligned implies $A\neq B$, $B\neq C$ and $A\neq C$.
\item A second one is $A,B,C$ not aligned implies that any different lines built from two of those points are intersecting.
\end{itemize}
\end{itemize}

And the algorithm would be
\begin{itemize}
\item Do we have the result umong the plane relations.
\item If yes, we can return the result.
\item Otherwise, thanks to the rules, we can build new relations.
\end{itemize}

The difficulty is the capability to mix the rules to achieve a result.

Let's return to the question: \textit{Are $(AB)$ and $(AC)$ parallel ?}

If we talk about those lines, we must be able to build them. The plane does not contain anything about the fact that $A$ and $B$ are distinct. So rules must be used.

Are there some rules returning the expected goal: $A\neq B$ ?

We could have several ones. So can take the corresponding rules.

What are each rules hypothesis ?

And then, we have the good rule telling that $A,B,C$ not aligned.

If rules gives the result, I think a good idea would be to add relations in the plane, which could make easier next questions.


\section{Rules}

Rules can be seen like functions.

We have seen a first one that is : $A,B,C$ not aligned $=>$ $A\neq B$, $B\neq C$ and $A\neq C$.

Relations are not binary, they work with an operator and a list of parameters so we can get them by their positions.

Then relations can be stored in a stack. The root of the stack contains current relations. When we call a rule a new step is added to the stack with results inside.

\textit{Stack} is not the best terminology. If we keep the multithreaded engine, it can contains several step at the same level.

The parser must understand a rule. We could use the following syntax:

\begin{verbatim}
Rule: A,B,C Not Aligned => A!=B, A!=C, B!=C
\end{verbatim}

We could store this as:
\begin{verbatim}
Rule: #1,#2,#3 Not Aligned => #1!=#2, #1!=#3, #2!=#3
\end{verbatim}

The first part of the rule is a big condition based on expressions we can see beside \texttt{Assume}. Its result is a boolean. If it is true, we have the result relations separated by commas.

Suppose we want to prove $A\neq B$.

If we have access to results, we can find \texttt{\#1!=\#2} and also \texttt{\#1!=\#3} and \texttt{\#2!=\#3}. By replacing \texttt{\#N} by the good letter, we obtain three possibilities.

We get then something to search among hypothesis like:
\begin{verbatim}
A,B,#3 Not Aligned
\end{verbatim}

That should be possible to find among relations stored in the plane.

Rules are stored in lists at the moment. It's not optimal but should do the work.
A rule is built of two parts:
\begin{itemize}
\item hypothesis
\begin{itemize}
\item The hypothesis has a \textbf{pointer to the rule itself}. This allows to go and back between hypothesis, rule and results.
\item The second point here is to store the condition. It can be simple, at first it is a relation with a list of arguments. Then, we can have two expressions more complicated, that are functions \texttt{and} and \texttt{or} that take as arguments others expressions.
\end{itemize}
\item result
\begin{itemize}
\item The result has a \textbf{pointer to the rule itself}. This allows to go and back between hypothesis, rule and results.
\end{itemize}
\end{itemize}

\section{Parser}

Because of rule, we have two steps.

Imagine the two following lines:
\begin{verbatim}
Assume A,B,C Aligned
Rule: A,B,C Aligned => A!=B, A!=C, B!=C
\end{verbatim}

The \textit{assumed part} and the rule's \textit{condition} are parsed the same way. So they give the same result, in a first step anyway.

A condition is stored as follows:

Elements are stored in an array, like this they are accessible by their index. Then, we have the expression with integers telling which is the argument.

For $A,B,C$ Aligned, we have:
\begin{itemize}
\item $[A, B, C]$
\item Aligned($0, 1, 2$)
\end{itemize}

For $A\neq B$ and $B \neq C$, we have:
\begin{itemize}
\item $[A,B,C]$
\item And(Distinct($0,1$),Distinct($1,2$))
\end{itemize}

Used functions return a boolean for now.

With such a solution, it is not easy to store the fact, $A\neq B$ and $B\neq C$ and check them with conditions. We must find a more efficient solution.

The present implementation, contains relations with an operator and a parameters list. It looks quite good except the operator name that should become a string and not an enum.

Now, the problem is when the user enter something like this

\begin{verbatim}
Assume A!=B And B!=C
\end{verbatim}

And a worse possibility is

\begin{verbatim}
Assume A!=B Or B!=C
\end{verbatim}

Relations between elements cannot be stored by relation name. We must store the boolean expression and evaluate it.

We also need something to evaluate those expressions.

When an expression is assumed, we should reduce it.

When we want to check if an expression is true, typically needed to verify a rule, it is difficult to directly verify the condition.

We have then two kinds of boolean relations:
\begin{enumerate}
\item Elementary relations, constructed only from geometric datas, for example $A\neq B$, $A, B, C$ aligned, etc\dots
\item Composed relations, using previous ones but also boolean expressions based on others boolean expressions, in fact when we use \textit{And}, \textit{Not}, \textit{Or}, \dots
\end{enumerate}

Rules or relations are boolean expressions, simple or complex. Some of these relations are facts and others are rules. So, the plane can contains to sets of expressions:
\begin{enumerate}
\item facts
\item rules: Let's remember that $A => B$ is equivalent to $!A \hbox{ or } B$
\end{enumerate}

Be careful! A fact is based on existing elements whereas a rule is built on anonymous objects. We must be able to replace anonymous variables by real ones when needed.

Let's continue with our example. $A,B,C$ are not aligned.

There is a rule saying that if $A,B,C$ are not aligned, then $A\neq B$ and idem for $B,C$ and $A,C$.

We want to know if $(AB)$ and $(AC)$ are parallel. Before answering the question, we must know if $(AB)$ exists. The condition is $A\neq B$.

Among the rules, do we have a conclusion containing \textit{Distinct(X,Y)}?

Yes, we have \texttt{X,Y,Z Not Aligned => X!=Y And Y!=Z And X!=Z}.

And There is an hypothesis telling that \texttt{A,B,C Not Aligned}.

The last question is \textit{can we mix}:
\begin{itemize}
\item $A,B,C Not Aligned$
\item \texttt{X,Y,Z Not Aligned => X!=Y And Y!=Z And X!=Z}
\item $A\neq B$
\end{itemize}
to prove that $A\neq B$.

The first example looks too simple now. Let's find a second one more representative.

Here is a new exercise:

$(AB)$ is a line. $(AC)$ and $(BD)$ are lines orthogonal to $(AB)$.
Let $\cal{D}$ be a line orthogonal to $(AC)$ or orthogonal to $(BD)$.
Is $\cal D$ parallel to $(AB)$?

We have several rules:
\begin{enumerate}
\item $(XY)$ line $=>$ $X\neq Y$
\item $\cal D$ orthogonal to $\cal D'$ and $\cal D$ parallel to $\Delta$ implies $\cal D'$ orthogonal to $\Delta$.
\item $\cal D$ orthogonal to $\cal D'$ and $\cal D'$ orthogonal to $\cal D"$ implies $\cal D$ parallel to $\cal D"$.
\end{enumerate}

Facts are:
\begin{enumerate}
\item \texttt{Ortho((AC), (AB))}.
\item \texttt{Ortho((BD), (AB))}.
\item \texttt{Or(Ortho($\cal D$, (AC)), Ortho($\cal D$, (BD)))}
\end{enumerate}


\section{Classes}
\subsection{DPBoolSimpFunc}

This class has
\begin{itemize}
\item A name
\item A list of parameters that are \textit{DPElements}
\end{itemize}

\subsection{DPBoolBoolFunc}

This class has
\begin{itemize}
\item A name
\item A list of parameters that are \textit{booleans}. The goal is to mix \textit{DPBoolSimpFuncs}.
\end{itemize}

\subsection{DPPointsSet}

This class is a points set. It is composed of one, two, \dots or an infinite number of points.

It has a name.

In the case of a finite set, it contains their list.

In the case of an infinite set, it must contains rules defining how those points are disposed.

To define a set, we can need points to define the set. Those points are not necessarily in the set. We have examples like \textit{circles}, \textit{mediatrices}, \textit{ellipses} which are defined from points not part of them.

To continue the previous remark, we can need numbers to define a set. It is the case for a circle (the radius), it is also the case for a segment (where we could want to define its length).

A last case, is that a set can be an union, an intersection of others sets.

Examples that come to mind are :

\begin{enumerate}
\item A line $\Delta$ is an infinite set of points. It can contains two points specified $A$ and $B$ and one rule could say "all points aligned with $A$ and $B$".
\item A circle $C$ is an infinite set of points. Its content is represented by all the points whose distances with a given point $A$ are the same: $r$.
\end{enumerate}


\begin{itemize}
\item \textit{name} - The name of the set.
\item \textit{list<DPPoint> content} - The list of points definining the set. Maybe this last one has others points. It is the case for a line ; it is usually defined by two points but has an infinite number of points.
\end{itemize}

\subsubsection{DPLine}

It is a subclass of \texttt{DPPointsSet}. It represents a \textit{line}.

With a such class, it is easier to recognize all its properties.

\subsubsection{DPCircle}

It is a subclass of \texttt{DPPointsSet}. It represents a \textit{line}.

\end{document}
